\documentclass[10pt]{article}
\usepackage[left=0.4in,right=0.4in,top=0.7in,bottom=0.4in]{geometry}
\usepackage{hyperref}
\usepackage{fancyhdr}
\usepackage{listings}
\usepackage{xcolor}
\usepackage{tocloft}
\usepackage{pdflscape}
\usepackage{multicol}
\usepackage{inconsolata}

\renewcommand*{\ttdefault}{pcr}
\renewcommand\cftsecfont{\fontsize{8}{9}\bfseries}
\renewcommand\cftsecpagefont{\fontsize{8}{9}\mdseries}
\renewcommand\cftsubsecfont{\fontsize{5}{6}\mdseries}
\renewcommand\cftsubsecpagefont{\fontsize{5}{6}\mdseries}
\renewcommand\cftsecafterpnum{\vspace{-1ex}}
\renewcommand\cftsubsecafterpnum{\vspace{-1ex}}

\lstdefinestyle{shared}{
    belowcaptionskip=1\baselineskip,
    breaklines=true,
    xleftmargin=\parindent,
    showstringspaces=false,
    basicstyle=\fontsize{9}{6}\ttfamily,
}
\lstdefinestyle{cpp}{
	style=shared,
    language=C++,
    keywordstyle=\bfseries\color{green!40!black},
    commentstyle=\itshape\color{red!80!black},
    identifierstyle=\color{blue},
    stringstyle=\color{purple!40!black},
}
\lstdefinestyle{java}{
    style=shared,
    language=Java,
    keywordstyle=\bfseries\color{green!40!black},
    commentstyle=\itshape\color{purple!40!black},
    identifierstyle=\color{blue},
    stringstyle=\color{orange},
}
\lstdefinestyle{py}{
    style=shared,
    language=Python,
    keywordstyle=\bfseries\color{green!40!black},
    commentstyle=\itshape\color{purple!40!black},
    identifierstyle=\color{blue},
    stringstyle=\color{orange},
}
\lstdefinestyle{txt}{
    style=shared,
}
\lstset{escapechar=@}

\pagestyle{fancy}
\fancyhead[L]{Universidad Nacional de Colombia}
\fancyhead[R]{\thepage}
\fancyfoot[C]{}

\fancypagestyle{plain}
{
\fancyhead[L]{Universidad Nacional de Colombia}
\fancyhead[R]{\thepage}
\fancyfoot[C]{}
}

\title{\vspace{-4ex}\Large{UNAL ICPC Team Notebook (2024)}}
\author{hola mUNdo}
\date{}

\begin{document}
\begin{landscape}
\begin{multicols}{2}

\maketitle
\vspace{-13ex}
\tableofcontents
\pagestyle{fancy}

\input contents.tex

\section{algorithm}%%%%%%%%%%%%%%%%%%ALGORITHM%%%%%%%%%%%%%%%%%%
\#include $<$algorithm$>$ \#include $<$numeric$>$ \\
\begin{tabular}{|l|l|p{5.4cm}|} \hline
\textbf{Algo} & \textbf{Params} &  \textbf{Funcion} \\  \hline
%swap & e1, e2 &  da vuelta e1,e2 & $1$\\\hline
sort, stable\_sort & f, l &  ordena el intervalo \\  \hline
%is\_sorted & f, l &  \textit{bool} si esta ordenado \\  \hline
nth\_element & f, nth, l & \textit{void} ordena el n-esimo, y \\ && particiona el resto \\  \hline
fill, fill\_n & f, l / n, elem & \textit{void} llena [f, l) o [f, \\ && f+n) con elem \\  \hline
lower\_bound, upper\_bound & f, l, elem & \textit{it} al primer / ultimo donde se \\ && puede insertar elem para que\\ && quede ordenada \\  \hline
binary\_search & f, l, elem & \textit{bool} esta elem en [f, l) \\  \hline
copy & f, l, resul & hace resul+$i$=f+$i$ $\forall i$ \\  \hline
find, find\_if, find\_first\_of & f, l, elem & \textit{it} encuentra i $\in$[f,l) tq. i$=$elem, \\ & / pred / f2, l2 & pred(i), i$\in$[f2,l2)\\\hline
count, count\_if & f, l, elem/pred & cuenta elem, pred(i)\\\hline
search & f, l, f2, l2 & busca [f2,l2) $\in$ [f,l)\\\hline
replace, replace\_if & f, l, old & cambia old / pred(i) por new \\ & / pred, new &\\\hline
reverse & f, l & da vuelta\\\hline
partition, stable\_partition & f, l, pred & pred(i) ad, !pred(i) atras\\\hline
%min, max & e1, e2 & men / may & $1$\\\hline
min\_element, max\_element & f, l, [comp] & \textit{it} min, max de [f,l]\\\hline
lexicographical\_compare & f1,l1,f2,l2 & \textit{bool} con [f1,l1]<[f2,l2]\\\hline
next/prev\_permutation & f,l & deja en [f,l) la perm sig, ant\\\hline
set\_intersection, & f1, l1, f2, l2, res & [res, $\ldots$) la op. de conj\\
set\_difference, set\_union, & & \\
set\_symmetric\_difference, & &\\\hline
push\_heap, pop\_heap, & f, l, e / e / & mete/saca e en heap [f,l), \\
make\_heap & & hace un heap de [f,l)\\\hline
is\_heap & f,l & \textit{bool} es [f,l) un heap\\\hline
accumulate & f,l,i,[op] & \textit{T} $=$ $\sum$/oper de [f,l)\\\hline
inner\_product & f1, l1, f2, i & \textit{T} $=$ i $+$ [f1, l1) . [f2, $\ldots$ )\\\hline
partial\_sum & f, l, r, [op] & r+i = $\sum$/oper de [f,f+i] $\forall i \in$[f,l)\\\hline
%power & e, i, op & \textit{T} = $e^{n}$\\\hline
\_\_builtin\_ffs& unsigned int & Pos. del primer 1 desde la derecha\\\hline
\_\_builtin\_clz & unsigned int & Cant. de ceros desde la izquierda.\\\hline
\_\_builtin\_ctz & unsigned int & Cant. de ceros desde la derecha.\\\hline
\_\_builtin\_popcount & unsigned int & Cant. de 1’s en x.\\\hline
\_\_builtin\_parity & unsigned int & 1 si x es par, 0 si es impar.\\\hline
\_\_builtin\_XXXXXXll & unsigned ll & = pero para long long's.\\\hline
\end{tabular}\newpage
\section{Math}%%%%%%%%%%%%%%%%%%MATH%%%%%%%%%%%%%%%%%%%%%%%%%%%%%%%%
\subsection{Identidades}
{
$\sum_{i=0}^n\binom{n}{i}=2^n$

$\sum_{i=0}^n i\binom{n}{i}=n*2^{n-1}$

$\sum_{i=m}^n i = \frac{n(n+1)}{2} - \frac{m(m-1)}{2} = \frac{(n+1-m)(n+m)}{2}$

$\sum_{i=0}^n i = \sum_{i=1}^n i = \frac{n(n+1)}{2}$

$\sum_{i=0}^n i^2 = \frac{n(n+1)(2n+1)}{6} = \frac{n^3}{3} + \frac{n^2}{2} + \frac{n}{6}$

$\sum_{i=0}^n i(i-1) = \frac{8}{6}(\frac{n}{2})(\frac{n}{2}+1)(n+1)$ (doubles) $\rightarrow$ Sino ver caso impar y par

$\sum_{i=0}^n i^3 = \left(\frac{n(n+1)}{2}\right)^2 = \frac{n^4}{4} + \frac{n^3}{2} + \frac{n^2}{4} = \left[\sum_{i=1}^n i\right]^2$

$\sum_{i=0}^n i^4 = \frac{n(n+1)(2n+1)(3n^2+3n-1)}{30} = \frac{n^5}{5} + \frac{n^4}{2} + \frac{n^3}{3} - \frac{n}{30}$

$\sum_{i=0}^n i^p = \frac{(n+1)^{p+1}}{p+1} + \sum_{k=1}^p\frac{B_k}{p-k+1}{p\choose k}(n+1)^{p-k+1}$

$r=e-v+k+1$

Teorema de Pick: (Area, puntos interiores y puntos en el borde)

$A=I+\frac{B}{2}-1$


}%
\subsection{Ec. Caracteristica}
$a_0T(n)+a_1T(n-1)+...+a_kT(n-k)=0$

$p(x)=a_0 x^k + a_1 x^{k-1} + ... + a_k$

Sean $r_1,r_2,...,r_q$ las raíces distintas, de mult. $m_1, m_2, ..., m_q$

$T(n)=\sum_{i=1}^q{\sum_{j=0}^{m_i - 1}c_{ij} n^j r_i^n}$

\subsection{Tablas y cotas (Primos, Divisores, Factoriales, etc)}
%\subsubsection{
\paragraph{Factoriales} \ \\
\begin{tabular}{l|l}
0! =	1             & 11! = 39.916.800  \\
1! =	1             & 12! =	479.001.600	($\in \mathtt{int}$)\\
2! =	2             & 13! =	6.227.020.800	\\
3! =	6             & 14! =	87.178.291.200	\\
4! =	24            & 15! =	1.307.674.368.000	\\
5! =	120   			  & 16! =	20.922.789.888.000	\\
6! =	720           & 17! =	355.687.428.096.000	\\
7! =	5.040	        & 18! =	6.402.373.705.728.000	\\
8! =	40.320	      & 19! =	121.645.100.408.832.000	\\
9! =	362.880       & 20! =	2.432.902.008.176.640.000	($\in \mathtt{tint}$) \\
10! =	3.628.800     & 21! =	51.090.942.171.709.400.000
\end{tabular}
 
max signed tint = 9.223.372.036.854.775.807 \\
max unsigned tint = 18.446.744.073.709.551.615

\paragraph{Primos cercanos a $10^n$}\ \\
9941 9949 9967 9973 10007 10009 10037 10039 10061 10067 10069 10079\\
99961 99971 99989 99991 100003 100019 100043 100049 100057 100069\\
999959 999961 999979 999983 1000003 1000033 1000037 1000039\\
9999943 9999971 9999973 9999991 10000019 10000079 10000103 10000121\\
99999941 99999959 99999971 99999989 100000007 100000037 100000039 100000049\\
999999893 999999929 999999937 1000000007 1000000009 1000000021 1000000033
 
\paragraph{Cantidad de primos menores que $10^n$}\ \\
$\pi(10^1)$ = 4 ;
$\pi(10^2)$ = 25 ;
$\pi(10^3)$ = 168 ;
$\pi(10^4)$ = 1229 ;
$\pi(10^5)$ = 9592 \\
$\pi(10^6)$ = 78.498 ;
$\pi(10^7)$ = 664.579 ;
$\pi(10^8)$ = 5.761.455 ;
$\pi(10^9)$ = 50.847.534 \\
$\pi(10^{10})$ = 455.052,511 ;
$\pi(10^{11})$ = 4.118.054.813 ;
$\pi(10^{12})$ = 37.607.912.018% ;
%
% Fuente: http://primes.utm.edu/howmany.shtml#table
%
%

%\subsubsection{Divisores}
\paragraph{Divisores} \ \\
Cantidad de divisores ($\sigma_0$) para \emph{algunos} $n / \neg\exists n'<n, \sigma_0(n') \geqslant \sigma_0(n)$ \\
$\sigma_0(60)$ = 12 ; $\sigma_0(120)$ = 16 ; $\sigma_0(180)$ = 18 ; $\sigma_0(240)$ = 20 ; $\sigma_0(360)$ = 24 \\
$\sigma_0(720)$ = 30 ; $\sigma_0(840)$ = 32 ; $\sigma_0(1260)$ = 36 ; $\sigma_0(1680)$ = 40 ; $\sigma_0(10080)$ = 72 \\ $\sigma_0(15120)$ = 80 ; $\sigma_0(50400)$ = 108 ; $\sigma_0(83160)$ = 128 ; $\sigma_0(110880)$ = 144 \\
$\sigma_0(498960)$ = 200 ; $\sigma_0(554400)$ = 216 ; $\sigma_0(1081080)$ = 256 ; $\sigma_0(1441440)$ = 288  $\sigma_0(4324320)$ = 384 ; $\sigma_0(8648640)$ = 448
 
%
Suma de divisores ($\sigma_1$) para \emph{algunos} $n / \neg\exists n'<n, \sigma_1(n') \geqslant \sigma_1(n)$ \\
$\sigma_1(96)$ = 252 ; $\sigma_1(108)$ = 280 ; $\sigma_1(120)$ = 360 ; $\sigma_1(144)$ = 403 ; $\sigma_1(168)$ = 480 \\
$\sigma_1(960)$ = 3048 ; $\sigma_1(1008)$ = 3224 ; $\sigma_1(1080)$ = 3600 ; $\sigma_1(1200)$ = 3844 \\
$\sigma_1(4620)$ = 16128 ; $\sigma_1(4680)$ = 16380 ; $\sigma_1(5040)$ = 19344 ; $\sigma_1(5760)$ = 19890 \\
$\sigma_1(8820)$ = 31122 ; $\sigma_1(9240)$ = 34560 ; $\sigma_1(10080)$ = 39312 ; $\sigma_1(10920)$ = 40320 \\
$\sigma_1(32760)$ = 131040 ; $\sigma_1(35280)$ = 137826 ; $\sigma_1(36960)$ = 145152 ; $\sigma_1(37800)$ = 148800 \\
$\sigma_1(60480)$ = 243840 ; $\sigma_1(64680)$ = 246240 ; $\sigma_1(65520)$ = 270816 ; $\sigma_1(70560)$ = 280098 \\
$\sigma_1(95760)$ = 386880 ; $\sigma_1(98280)$ = 403200 ; $\sigma_1(100800)$ = 409448  \\
$\sigma_1(491400)$ = 2083200 ; $\sigma_1(498960)$ = 2160576 ; $\sigma_1(514080)$ = 2177280 \\
$\sigma_1(982800)$ = 4305280 ; $\sigma_1(997920)$ = 4390848 ; $\sigma_1(1048320)$ = 4464096 \\
$\sigma_1(4979520)$ = 22189440 ; $\sigma_1(4989600)$ = 22686048 ; $\sigma_1(5045040)$ = 23154768 \\
$\sigma_1(9896040)$ = 44323200 ; $\sigma_1(9959040)$ = 44553600 ; $\sigma_1(9979200)$ = 45732192
%
%

\end{multicols}
\end{landscape}
\end{document}